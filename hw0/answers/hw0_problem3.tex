\documentclass[11pt]{article}
\usepackage[shortlabels]{enumitem}
\usepackage[margin=1in,headheight=15pt]{geometry}   % Adjusted headheight
\usepackage{amsmath}
\usepackage{fancyhdr}
\usepackage{graphicx}
\usepackage{cancel}
\usepackage{amsfonts}
\usepackage{setspace}

% Set up fancy header/footer
\pagestyle{fancy}
\fancyhead[LO,L]{Jimmy Chen}
\fancyhead[CO,C]{CSCI 2600 - Principles of Software}
\fancyhead[RO,R]{November 13, 2023}
\fancyfoot[LO,L]{}
\fancyfoot[CO,C]{\thepage}
\fancyfoot[RO,R]{}
\renewcommand{\headrulewidth}{0.4pt}
\renewcommand{\footrulewidth}{0.4pt}
\onehalfspacing

\begin{document}
\section{Problem 3}

\subsection{Why did Fibonacci fail the testThrowsIllegalArgumentException test? What did you have to do to fix it?}
\textbf{Answer:}
The testThrowsIllegalArgumentException failed because when n was 0, it would instantly throw the exception, even though 0 is valid since it is a non-negative. To fix it, 
I changed the if statement from \texttt{if (n <= 0)} to \texttt{if (n < 0)}.

\subsection{Why did Fibonacci fail the testBaseCase test? What (if anything) did you have to doto fix it?}
\textbf{Answer:}
The fibonacci failed the testBaseCase because of the same issue in the testThrowsIllegalArgumentException, where the 0 is being thrown instead of outputting 0. 
The test passed after I fixed the first error.

\subsection{Why did Fibonacci fail the testInductiveCase test? What (if anything) did you have to do to fix it?}
\textbf{Answer:}
The Fibonacci failed the testInductiveCase test because when n was 2, because it was returning itself, when it should've went to the else statement. It also didn't work because
in the else statement it was getFibTerm(n+1)-getFibTerm(n-2), and fibonnaci is supposed to be the the sum of the previous two numbers. 
To fix the errors, I edited the else if statment to change it to n < 2, and also fixed the else statement so that it was getFibTerm(n-1)+getFibTerm(n-2).

\subsection{Why did Fibonacci fail the testLargeN test? What (if anything) did you have to do to fix it?}
\textbf{Answer:}
The fibonacci fail the testLargeN test because it was taking too long to compute the number.
To fix it I had to make it faster by using a list to store the previous numbers, and then adding the previous two numbers to get the next number. The number was also too large, so I had to change the ints to longs.

\subsection{What was causing Fibonacci to be so slow on testLargeN test? What did you do to make Fibonacci faster while still preserving the recursive nature of your implementation?}
\textbf{Answer:}
The reason why the fibonacci was so slow was because it was calling itself multiple times, and it was also calling the same numbers multiple times. To make it faster, I used a list to store the previous numbers, and then adding the previous two numbers to get the next number.
In the list I also added the first two numbers, 0 and 1, so that it would be able to add the previous two numbers. I also changed the ints to longs because the number was too large.
\end{document}