\documentclass[11pt]{article}
\usepackage[shortlabels]{enumitem}
\usepackage[margin=1in,headheight=15pt]{geometry}   % Adjusted headheight
\usepackage{amsmath}
\usepackage{fancyhdr}
\usepackage{graphicx}
\usepackage{cancel}
\usepackage{amsfonts}
\usepackage{setspace}

% Set up fancy header/footer
\pagestyle{fancy}
\fancyhead[LO,L]{Jimmy Chen}
\fancyhead[CO,C]{CSCI 2600 - Principles of Software}
\fancyhead[RO,R]{February 29, 2024}
\fancyfoot[LO,L]{}
\fancyfoot[CO,C]{\thepage}
\fancyfoot[RO,R]{}
\renewcommand{\headrulewidth}{0.4pt}
\renewcommand{\footrulewidth}{0.4pt}
\onehalfspacing

\begin{document}
\section{Problem 3}

\subsection{Write a pseudocode algorithm for polynomial division. Write your answer in the file
answers/problem3.pdf.}
\begin{verbatim}

//precondition: u and v are polynomials && u and v are not null
    && v != 0 && degree(u) >= degree(v)
function div(u, v)
    if degree(v) > degree(u) then
        return u / v
    q = 0
    r = u
    while degree(r) >= degree(v) do
        t = c * x^(degree(r) - degree(v))
        q = q + t
        r = r - t * v
    end while
    return q
end function
//postcondition: u = qv

\end{verbatim}

\subsection{When writing pseudocode use symbols +, -, *, and / to express rational
number and polynomial arithmetic. You may also use u[i] to retrieve the
coefficient at power i of polynomial u, as well as c * $x^{i}$
to denote the single-term polynomial of degree i and coefficient c.}
\begin{verbatim}
//precondition: u and v are polynomials && u and v are not null
    && v != 0 && degree(u) >= degree(v)
function div(u, v)
    if degree(v) > degree(u) then
        return u / v
    q = 0
    r = u
    while degree(r) >= degree(v) do
        c = r[degree(r)] / v[degree(v)]
        t = c * x^(degree(r) - degree(v))
        q = q + t
        r = r - t * v
    end while
    return q
end function
//postcondition: u = qv
\end{verbatim}

\subsection{State the loop invariant for the main loop and prove partial correctness. Write your answer in
the file answers/problem3.pdf. For the proof question, you do not need to handle division
by zero; however, you will need to do so in the Java program.
Important: write your pseudocode, invariants, and proofs first, then write the Java code.
Going backwards will be harder.}

loop invariant: $u = qv + r$ // where $r$ is the remainder of $u$ divided by $v$ and $q$ is the quotient of $u$ divided by $v$.\\
precondition: u and v are polynomials \&\&\ u and v are not null
\&\& v != 0 \&\& degree(u) $>=$ degree(v)\\
postcondition: u == qv\\

\noindent Proof: \\
\indent Base Case: \\
Before the loop, $u = qv + r$ where $q = 0$ and $r = u$.\\
Given q = 0\\
Assume r != 0, then degree(r) $>=$ degree(v)\\
V*0 = 0, so r $>=$ 0\\

Inductive Step: \\
Assume $u = qv + r$ holds at itr k\\
$q\_new = q\_old + t$\\
$r\_new = r\_old - t * v$\\
This gives us:\\
$u\_new = q\_new * v + r\_new$\\
$u\_new = (q\_old + t)v + (r\_old - t * v\_old)$\\
$u\_new = q\_old * v + t * v + r\_old - t * v$\\
$u\_new = q\_old * v + r\_old$\\

This proves that the loop invariant holds at each iteration k.\\

\noindent Implication Post Condition: \\
u ==qv + r and !((degree(r) $>=$ degree(v))\\
u == qv + r \&\& r == 0, thus u == qv, implying the post condition\\

\end{document}